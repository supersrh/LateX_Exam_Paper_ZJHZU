% !Mode::"TeX:UTF-8"
\documentclass[twocolumn,landscape,UTF8]{article}
\usepackage{listings,lstautogobble}
\usepackage{booktabs}
\usepackage{verbatim}
\usepackage{ctex}
\usepackage{lastpage}
%\usepackage{times} %use the Times New Roman fonts
\usepackage{color}
%\usepackage{placeins}
%------------------------------------------------------------------------------
\usepackage{lastpage}%获得总页数
%------------------------------------------------------------------------------
\usepackage{ulem}
\usepackage{titlesec}
\usepackage{graphicx}
\usepackage{colortbl}
\usepackage{listings}
\usepackage{makecell}
\usepackage{indentfirst}
\usepackage{fancyhdr}
\usepackage{setspace} % 行间距
\usepackage{bm}%\boldsymbol 粗体
% 数学
\usepackage{amsmath,amsfonts,amsmath,amssymb,times}
\usepackage{txfonts}
\usepackage{enumerate}% 编号
\usepackage{tikz,pgfplots} %绘图
\usepackage{tkz-euclide,pgfplots}
\usetikzlibrary{automata,positioning}
%\usepackage[paperwidth=18.4cm,paperheight=26cm,top=1.5cm,bottom=2cm,right=2cm]{geometry} % 单页
\usepackage[paperwidth=36.8cm,paperheight=26cm,top=1.5cm,bottom=2.0cm,right=1cm]{geometry}
\lstset{language=C,keywordstyle=\color{red},showstringspaces=false,rulesepcolor=\color{green}}
\oddsidemargin=0.1cm   %奇数页页边距
\evensidemargin=0.1cm %偶数页页边距
%\textwidth=14.5cm        %文本的宽度 单页
\textwidth=34cm        %文本的宽度 单页
%------------------------------------------------------------------------------------------------------------------------------
\newsavebox{\zdx}%装订线
\newcommand{\putzdx}{\marginpar{
		\parbox{1cm}{\vspace{-2.5cm}         %%--调整装订线的上下高度
			\rotatebox[origin=c]{90}{        %%--设置左侧装订线内内容在文章中的倾斜程度,“90”就是逆时针旋转90度
				\usebox{\zdx}
		}}
}}
\newcommand{\blank}{\uline{\textcolor{white}{a}\ \textcolor{white}{a}\ \textcolor{white}{a}\ \textcolor{white}{a}\ \textcolor{white}{a}\ \textcolor{white}{a}\ \textcolor{white}{a}\ \textcolor{white}{a}\ \textcolor{white}{a}\ \textcolor{white}{a}\ \textcolor{white}{a}}}
\newcommand{\me}{\mathrm{e}}  %定义 对数常数e,虚数符号i,j以及微分算子d为直立体。
\newcommand{\mi}{\mathrm{i}}
\newcommand{\mj}{\mathrm{j}}
\newcommand{\dif}{\mathrm{d}}
\newcommand{\bs}{\boldsymbol}%数学黑体
\newcommand{\ds}{\displaystyle}
%通常我们使用的分数线是系统自己定义的分数线,即分数线的长度的预设值是分子或分母所占的最大宽度,如何让分数线的长度变长成,我们%可以在分子分母添加间隔来实现。如中文分式的命令可以定义为:
%\newcommand{\chfrac[2]}{\cfrac{\;#1\;}{\;#2\;}}
%\frac{1}{2} \qquad \chfrac{1}{2}
%选择题
\newcommand{\fourch}[4]{\\\begin{tabular}{*{4}{@{}p{3.5cm}}}(A)~#1 & (B)~#2 & (C)~#3 & (D)~#4\end{tabular}} % 四行
\newcommand{\twoch}[4]{\\\begin{tabular}{*{2}{@{}p{7cm}}}(A)~#1 & (B)~#2\end{tabular}\\\begin{tabular}{*{2}{@{}p{7cm}}}(C)~#3 &
		(D)~#4\end{tabular}}  %两行
\newcommand{\onech}[4]{\\(A)~#1 \\ (B)~#2 \\ (C)~#3 \\ (D)~#4}  % 一行
\renewcommand{\headrulewidth}{0pt}
\pagestyle{fancy}
%----------%%%---------%%%----------%%%---------%%%----------%%%-----------%%%-----------%%%---------
%%---------%%%---------%%%----------%%%---------%%%----------%%%----------%%%-----------%%%----------
\NewDocumentEnvironment{answer}{ +b }{
	\ifanswer
	{#1}
	\else
	\hphantom
	{#1}
	\fi
}{}
\newif\ifanswer

% answertrue %% Set  answertrue to show the answer environmentanswerfalse %% Set  answerfalse to hide the answer environment
\NewDocumentEnvironment{answerj}{ +b }{
	\ifanswerj
	{#1}
	\else
	\hphantom
	{#1}
	\vspace{2.5cm}
	\fi
}{}
\newif\ifanswerj

\NewDocumentEnvironment{answercalc}{ +b }{
	\ifanswercalc
	{#1}
	\else
	\hphantom
	{#1}
	\vspace{10cm}
	\fi
}{}
\newif\ifanswercalc


\answerfalse
\answerjfalse
\answercalcfalse
\answertrue
\answerjtrue
\answercalctrue

\lstset{
autogobble=true
}

\begin{document}
	\linewidth 14.5cm
	
	%---------------------------------------------------------------------------------------------------------------------------------------------------------------
	\fancyhf{}
	\begin{comment}
		\fancyfoot[CO,CE]{\vspace*{1mm}第\,\thepage\,页 , 共 ~\pageref{LastPage} 页}
		\sbox{\zdx}
		{\parbox{27cm}{
				\centering \CJKfamily{song}姓~名\underline{\makebox[34mm][c]{}} \CJKfamily{song}学~号\underline{\makebox[34mm][c]{}}~班~级\underline{\makebox[34mm][c]{}}~ 座号~\underline{\makebox[34mm][c]{}}\\
				\vspace{1.5mm}
				请把答案写在答题卡上,并在试卷上填写姓名、学号、班级、座号\\
				\vspace{0.5mm}
				\dotfill{} 装\dotfill{}订\dotfill{}线\dotfill{} \\
		}}
		\reversemarginpar
	\end{comment}
	
	\fancyfoot[CO,CE]{\vspace*{1mm}第\,\thepage\,页 , 共 ~\pageref{LastPage} 页}
	\sbox{\zdx}
	{\parbox{27cm}{ 
			\dotfill{} 装\dotfill{}订\dotfill{}线\dotfill{} \\
	}}
	\reversemarginpar
	%%-----------------------------------------------------------------------------------------------------------------------------------------------------------------
	\setlength{\marginparsep}{1.0cm}
	\putzdx                            %%装订线--奇页数
	%%-----------------------------------------------------------------------------------------------------------------------------------------------------------------
	\begin{center}
		\begin{LARGE}
			湖州学院~\underline{~2022~-- 2023 }\,学年第\,\underline{~1~}\,学期\\ 
			\vspace{2mm}{\bf 《XXXX》期末考试试卷}\\
		\end{LARGE}
		\vspace{2mm} *开卷考试,可带纸质资料,禁止使用\underline{计算器}以外的电子设备*\\
		\vspace{2mm}适用班级 \underline{~12345678~}~~~~~~考试时间 \underline{~90~} 分钟 \\
		\vspace{2mm}院(系) \underline{~~~~~~~~~~~~~~~~} ~~班级 \underline{~~~~~~~~~~~~~~~~} ~~学号 \underline{~~~~~~~~~~~~~~~~}~~姓名 \underline{~~~~~~~~~~~~}~~座位号 \underline{~~~~~~~~~~}
		\par
		\vspace{4mm}
		\begin{tabular}{|c|c|c|c|c|c|c|c|c|c|c|c|}
			\hline
			~题号~ & ~~一~~ & ~~二~~ & ~~三~~ & ~~四~~ & ~~五~~ & ~~六~ & ~~七~ & ~~八~ & ~~九~ & ~~十~ & ~总分~ \\
			\hline
			分数 &    &    &  & & & & & & & & \\
			\hline
		\end{tabular}
		
		%\vspace{0.5cm}
	\end{center}
	%\vspace{-0.5cm}
	
	%%-----------------%%%------------%%%--------------%%%-----------------%%%-------------%%%---------%%%---------%%%-----------分段:\par,或两个回车
	%%%%%--------------%%%-------------%%%-------------%%%-----------------%%%-------------%%%--------%%%---------%%%------------导引线:\dotfill,\hrulefill
	
	\section*{\begin{tabular}{|c|}
			\hline
			\normalsize \normalfont 得分 \\
			\hline
			\\
			\hline
		\end{tabular} ~~一、选择题~(每小题~2 分,共~20 分) }					
	
	\begin{enumerate}\setcounter{enumi}{0}
		%-----------------------------------------------------------------------------------
		\item QQQQQQQQQQQQQQQQQQQQQQQQQQQQQQQQ ~(~~\begin{answer}D\end{answer}~~)
		\onech{AAAAAAAAAAAAAAAAAAA}{BBBBBBBBBBBBBBBBBBBBBB}{CCCCCCCCCCCCCCCCCCCCCCC}{DDDDDDDDDDDDDDDDDDDDDD}
		
		\begin{comment}
			\item 概念设计的规模应是 ~(~~\begin{answer}B\end{answer}~~)
			\twoch{小试时的最佳规模}{工业化时的最佳规模}{中试时的最佳规模}{大型试验时的最佳规模}
		\end{comment}
		
		\item QQQQQQQQQQQQQQQQQQQQQ ~(~~\begin{answer}A\end{answer}~~)
		\twoch{AAAAAAAAAA}{BBBBBBBBBB}{CCCCCCCCCC}{DDDDDDDDDD}
		QQQQQQQQQQQQQQQQQQQQQQQQQQQQ
		\item QQQQQQQQQQQQQQQQQQQQQQQQQQQQ ~(~~\begin{answer}B\end{answer}~~)
		\fourch{AAAAA }{BBBBB}{CCCCC}{DDDDD}
		
		
		
		
		
		%----------------------------------------------------------------------------
	\end{enumerate}
	%%-----------%%%-------------%%%--------------%%%--------------%%%-------------%%%------------
	%%-----------%%%-------------%%%--------------%%%--------------%%%-------------%%%------------
	
	\vspace{0.3cm}
	\section*{\begin{tabular}{|c|}
			\hline
			\normalsize \normalfont 得分 \\
			\hline
			\\
			\hline
		\end{tabular} ~~二、填空题~(每小题~2 分, 共~ 20 分) }	
	
	
	\begin{enumerate}\setcounter{enumi}{10}
		%----------------------------------------------------------------------------------------------
		
		
		\item XXXXXXXXXXXXXXXXXXX: ~\underline{~~\begin{answer}XX\end{answer}~~},~\underline{~~\begin{answer}XXXXXX\end{answer}~~},~\underline{~~\begin{answer}XXXXXXXXXXX\end{answer}~~},~\underline{~~\begin{answer}XXX\end{answer}~~}。
		
		%---------------------------------------------------------------
	\end{enumerate}
	\vspace{0.3cm}		
	%%如果想另起一页,用“\newpage”
	%%\putzdx %%装订线--奇页数
	%%-------%%%------------%%%--------------%%%-------------%%%------------%%%-----------------%%%---------------------	
	%%------%%%-------------%%%---------------%%%------------%%%------------%%%-----------------%%%---------------------
	
	\section*{\begin{tabular}{|c|}
			\hline
			\normalsize \normalfont 得分 \\
			\hline
			\\
			\hline
		\end{tabular} ~~三、简答题~( 共~20 分) 
	}	
	
	\begin{enumerate}\setcounter{enumi}{20}
		%----------------------------------------------------------------------
		
		\item ~(5分)XXXXXXXXXXXXXXXXXXXx。
		
		\begin{answerj}
			(答案仅供参考,回答合理适当给分)\\
			坚持需求导向、标准导向、特色导向,以社会需求为前提,以专业认证标准为参照,强化专业特色,持续提升专业内涵和建设水平。要以专业认证标准促进专业高质量发展,落实“学生中心、产出导向、持续改进”的理念,建强用好基层教学组织,形成以提高人才培养水平为核心的质量文化。
			一流专业建设点要以新思想、新理念、新技术、新方法、新标准、新体系为引领,建设一批新工科、新医科、新文科示范性本科专业,建设一批适应创新型、复合型、应用型人才培养需要的一流本科课程,在专业改革创新、师资队伍、教学资源、质量保障体系等各方面发挥示范辐射作用。
		\end{answerj}
		
		\item ~(5分)模板使用。
		
		\begin{lstlisting}
			在代码第98行,
			\answertrue
			\answerjtrue
			\answercalctrue
			表示显示答案
			
			\answerfalse
			\answerjfalse
			\answercalcfalse
			表示隐藏答案
			
			隐藏答案后留白的空间可用代码块控制,如:
			\NewDocumentEnvironment{answercalc}{ +b }{
				\ifanswercalc
				{#1}
				\else
				\hphantom
				{#1}
				\vspace{10cm}
				\fi
			}{}
			\newif\ifanswercalc
			
			改变\vspace{}中的数值,
			即可改变以 answercalc 定义的答案的留白空间
			
			可自行定义\NewDocumentEnvironment灵活调整页面
		\end{lstlisting}
		

		
		
		%------------------------------------------------------------
	\end{enumerate}
	
	
	%%------%%%%---------%%%%-------%%%-------------%%%----------%%%----------%%%%-------------%%%------------%%%%-----------%%%%-------------
	%%-------%%%---------%%%---------%%%------------%%%-----------%%%----------%%%-------------%%%------------%%%-------------%%%-------------
	\vspace{2cm}
	\clearpage
	
\end{document}
